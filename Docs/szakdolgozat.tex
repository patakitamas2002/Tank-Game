%
% Szakdolgozatminta az Eszterházy Károly Katolikus Egyetem
% matematika illetve informatika szakos hallgatóinak.
%

\documentclass[
% opciók nélkül: egyoldalas nyomtatás, elektronikus verzió
% twoside,     % kétoldalas nyomtatás
% tocnopagenum,% oldalszámozás a tartalomjegyzék után kezdődik
]{thesis-ekf}
\usepackage[T1]{fontenc}
\PassOptionsToPackage{defaults=hu-min}{magyar.ldf}
\usepackage[magyar]{babel}
\usepackage{mathtools,amssymb,amsthm,pdfpages}

\footnotestyle{rule=fourth}

\newtheorem{tetel}{Tétel}[chapter]
\theoremstyle{definition}
\newtheorem{definicio}[tetel]{Definíció}
\theoremstyle{remark}
\newtheorem{megjegyzes}[tetel]{Megjegyzés}

\begin{document}

\institute{Matematikai és Informatikai Intézet}
\title{A szakdolgozat címe}
\author{Pataki Tamás\\Programtervező Informatikus BSc}
\supervisor{Tanár neve\\beosztás}
\city{Eger}
\date{2024}
\maketitle

\tableofcontents

\chapter*{Bevezetés}
\addcontentsline{toc}{chapter}{Bevezetés}
A szakdolgozati a téma választás idején sok lehetőség közül lehetett választanom.
Végül amikor választnom kellet olyan témát szerettem volna amely egy sajátos kiívást és egyben az érdekeltségi közömbe is tartozik téma szerint.
A játékfejlesztés érdekelt korábban is, 

Ezek a faktorok alapján választottam játék fejlesztést Unity\cite{Unity} játékmotorral, melyben a fejlesztész C\# programozási nyelven alapszik. A 3D projekt mellet döntöttem, ez felelt meg az elképzeléseimhez. 



\chapter{A játék bemutatása}


\section{A játékmenet}

A fő eleme a játéknak a Tank részeinek kiválasztásának a lehetősége, melyet a főmenüben lehetséges.
\subsection{Tank kiválasztása}

Egy tank 3db részre van osztva.
\begin{itemize}
    \item Test
    \item Torony
    \item Ágyú
\end{itemize}
Mindhárom résznek van súlya, páncélzata.

\subsection{Csata}



\section{A megvalósítás}

\subsection{A tank felépítése}

\subsubsection{Páncélzati rendszer}
Minden tank modellje fel van osztva több kis darabra, melynek van a beépített ütközés komponense és a saját páncél komponense. Ezzel a módszerrel pontosan meg lehet adni hogy mely részen mekkora védelme van.

\subsubsection{Irányítás}

\subsubsection{Célzás}

\subsubsection{Torony és annak forgatása}

A torony forgatását egy üres játék objektum segítségével megy.

\subsubsection{Cső és annak emelése}

A torony forgatásához hasonlóan a cső is egy üres játék objektum használ, mely felé

\subsubsection{Játékos nézőpontja}

A játékos nézőpontja 2 darab kamerával van megoldva, külső és belső nézetes, melyeknek az érzékenységét külön lehet beállítani.

\subsection{Menü}


\subsubsection{Tank kiválasztása}

A kiválasztási menüben a felső részen cserélhető és látható a neve a tank részeinek.

\subsubsection{Beállítások}

A beállításokban az érzékenységet és a hangerőt lehet állítani, melyeknek

\subsubsection{Szünet menü}

Egy egyszerűbb szünet menü van a játékban, mellyel újralehet indítani a játékmenetet vagy visszamenni a fő menübe. Ilyenkor a játékmenet teljesen megáll illetve a kurzor újra látható és mozgathatóvá válik.

\subsection{Irányítás bevitele}

A bevitelhez a Unity új beviteli rendszerét használtam segítségül.


\subsection{Lövedékek}

A lövedékek fizikai objektumként működnek,

Három féle lövedék van beimplementálva jelenleg a játékba:
\begin{itemize}
    \item páncéltörő
    \item robbanótöltetes pácéltörő
    \item robbanótöltet
\end{itemize}

Ezekből a változók beállításával lehet további típusokat beállítani.

\subsubsection{Páncéltörő lövedék}

A töltet nélküli páncél törő lövedék a DeMarr formula alapján lett modellezve. Ide rakd a formulát.

\subsubsection{Robbanó töltetes lövedék}

Az alábbi lövedék egy felülethez érvén egy robbanást olyan módon utánoz hogy a megadott átmérőjű gömbön belül először megnézni az összes objektum között hogy bármyelik páncélnak számít-e, majd a robbanás középpontjától számítva mely páncéllemeznek a legkevesebb a vastagsága és a távolságának a keveréke.

\subsubsection{Robbabótöltetes páncéltörő lövedék}

Ez a lövedék hasonlóan működik a

\subsection{Kezelő felület}

\subsubsection{Életerő csík}

Két féle életerő csík létezik a projektben. Az egyik a kezelő felületen fixen található a bal alsó sarokba, ez kijelzi a teljes és a jelenlegi életerőt számmal, illetve hiányzó életerő szerint kevesebb része lesz kitöltve. A másik az ellenfelek felett megtalálható, és a 3D térben helyezkednek ell

\subsubsection{Célkereszt}

A cél kereszt 2 részből áll. A kör alakú irányzék a képernyő közepét és a kívánt célpontot jelzi, emellett lövés után pirossá változik, jelezve az újratöltés menetét. A második, kereszt alakú, pedig azt jelzi, hogy a cső jelenleg milyen irányba néz.

\subsection{Ellenfelek}

\subsubsection{Navmesh}

Az ellenfelek útkeresését a Unity-nak a beépített Navmesh rendszere segítségével oldottam meg.

\subsubsection{Állapotgép}

Állaptgép használatával van megoldva az ellenfelek viselkedésének szétválasztása. Azért választottam állapotgépet ehhez, mert egy tiszta és egyszerűen bővíthető módszert ad különböző viselkedések és állapotok megadására.

Kétféle állapot van:
\begin{itemize}
 \item Járőrözés
 \item Támadás
\end{itemize}

A járőrözés

\chapter{Felhasznált technológiák}

\section{Unity}

A fejlesztésem alapjának a Unity-t választottam népszerűsége és sokoldalúsága miatt. Rengeteg eszközt ad a fejlesztő kezébe, így szinte bármilyen típusú projektet lehet vele készíteni, akár a játékfejlesztésen túl is.

\subsection{A Unity-ről általánosan}

A projektek Unity-ben több jelenetből (scene-ből) épülnek fel, melyek tartalmaznak Gameobjecteket. 

\subsection{Gameobject-ek}

Egyik alapeleme a Unity-nek a GameObjectek, melyeknek rengeteg féle felhasználási módjuk van. Bármilyen karakter, pálya vagy tárgy egy Gameobject egy jeleneten belül. 
Alapból nincsen konkrét funkciójúk, de komponenseket lehet hozzájuk csatolni, melyek meghatározzás a működésüket. Az egyik alap komponen a Transform, mely minden gameobject-nek van, és a pozíciót, forgatást és méretet kezeli. 



\subsection{Scene-k}


\subsection{Szkriptelés}

A szkriptek egyféle komponens a Gameobjectnek. A játék logikájának és interakcióinak megírására szolgálnak. A szkriptek főként C\# nyelven készülnek, és az egyes játékobjektumok viselkedését, mozgását, valamint a játékbeli eseményeket szabályozzák.


A Monobehavior-ből nevű osztályból öröklődik a legtöbb szkript, mely többféle életciklus metódust:

\begin{itemize}
    \item Start
    \item Update
    \item FixedUpdate
    \item OnCollisionEnter / OnTriggerEnter
\end{itemize}

A Start metódus egyszer fut le a szkript indításakor, amikor a jelenet betöltődik, és az objektum engedélyezve van. Általában inicializálási feladatokat, például változók értékeinek beállítását hajtja végre.
\\ 
Az Update metódus minden képkockánál lefut, így ide kerülnek azok a kódok, amelyeknek folyamatosan, valós időben kell futniuk, például kamerakövetés.

A FixedUpdate, az Update-hez hasonlóan, folyamatosan fut, viszont minden fizikai képkockán fut le, a megjelenített képkockáktől függetlenul. Ide a fizikai számítások kerülnek.

Az OnCollisionEnter akkor fut le amikor két tárgy amelyek van Collider-e összeütközik. Ezt 

\subsection{Prefab-ek}



\section{Blender}

A modellezéshez Blendert választottam a tág eszközkészlete és sok tanulási anyaga miatt.


\chapter{További tervek}

\chapter*{Összegzés}

\addcontentsline{toc}{chapter}{\bibname}

\begin{thebibliography}{55}
    \bibitem{Unity}
    \textsc{Unity} \emph{Unity Documentation}
    \\
    \url{https://docs.unity.com/}



\end{thebibliography}

\end{document}
